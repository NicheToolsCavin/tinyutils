\documentclass{article}
\usepackage{amsmath}
\usepackage{graphicx}
\usepackage{hyperref}

\title{Advanced LaTeX Features}
\author{TinyUtils Team}
\date{\today}

\begin{document}

\maketitle

\section{Introduction}

This is an expanded \LaTeX{} document demonstrating advanced features including mathematics, lists, figures, tables, and footnotes\footnote{This is the first footnote, testing note preservation in LaTeX conversion.}.

\section{Mathematical Equations}

\subsection{Display Equations}

The quadratic formula is given by:
\begin{equation}
x = \frac{-b \pm \sqrt{b^2 - 4ac}}{2a}
\label{eq:quadratic}
\end{equation}

Einstein's mass-energy equivalence:
\begin{equation}
E = mc^2
\end{equation}

\subsection{Inline Mathematics}

Inline math can be written like $E = mc^2$ or $\pi \approx 3.14159$. The Pythagorean theorem states that $a^2 + b^2 = c^2$ for right triangles\footnote{Named after the Greek mathematician Pythagoras.}.

\section{Lists and Structures}

\subsection{Itemized List}

\begin{itemize}
    \item First item
    \item Second item with \textbf{bold text}
    \item Third item with \textit{italic text}
    \item Fourth item with \texttt{monospace code}
    \begin{itemize}
        \item Nested sub-item A
        \item Nested sub-item B
    \end{itemize}
\end{itemize}

\subsection{Enumerated List}

\begin{enumerate}
    \item Introduction and background
    \item Methodology and approach
    \item Experimental results
    \item Discussion and analysis
    \item Conclusion and future work
\end{enumerate}

\section{Tables}

\begin{table}[h]
\centering
\begin{tabular}{|l|c|r|}
\hline
\textbf{Column A} & \textbf{Column B} & \textbf{Column C} \\
\hline
Left aligned & Centered & Right aligned \\
Data 1 & Data 2 & Data 3 \\
Row 3 & Value & 42 \\
\hline
\end{tabular}
\caption{Example table demonstrating alignment and borders.}
\label{tab:example}
\end{table}

See Table~\ref{tab:example} for an example of tabular data.

\section{Figures and Graphics}

Figure environments are commonly used in scientific documents\footnote{Figures allow for captions and cross-references, making documents more professional.}:

\begin{figure}[h]
\centering
% Note: In a real document, this would include an actual image
% \includegraphics[width=0.5\textwidth]{example-image}
\fbox{\parbox{0.4\textwidth}{\centering [Placeholder for Figure]\\Image would appear here}}
\caption{Placeholder figure demonstrating figure environment.}
\label{fig:example}
\end{figure}

As shown in Figure~\ref{fig:example}, figures can be referenced in text.

\section{Code and Verbatim Text}

Inline code: \texttt{print("Hello World")}

Code block example:
\begin{verbatim}
def factorial(n):
    if n <= 1:
        return 1
    return n * factorial(n - 1)
\end{verbatim}

\section{Cross-References}

LaTeX supports cross-referencing equations like Equation~\ref{eq:quadratic}, tables like Table~\ref{tab:example}, and figures like Figure~\ref{fig:example}.

\section{Conclusion}

This document demonstrates comprehensive \LaTeX{} features: sections, equations, lists (nested and flat), tables, figures, code blocks, and footnotes. These elements should be preserved during format conversion to ensure fidelity\footnote{Fourth and final footnote for testing multiple notes in LaTeX.}.

\end{document}
